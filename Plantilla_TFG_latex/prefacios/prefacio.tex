\input{portada/portada_2}


\cleardoublepage
%\thispagestyle{empty}

\begin{center}
{\large\bfseries Aplicación para la gestión de una tienda minorista}\\
\end{center}
\begin{center}
Julia María Cano Flores\\
\end{center}

\vspace{0.7cm}
\noindent{\textbf{Palabras clave}: minorista, base de datos, software, requisitos}\\

\vspace{0.7cm}
\noindent{\textbf{Resumen}}\\

El desarrollo de este Trabajo Fin de Grado (TFG) tiene como finalidad principal simplificar y mejorar la eficiencia de la gestión de negocios minoristas. Está diseñado para optimizar el proceso de seguimiento de ventas mediante una base de datos que registre todas las transacciones realizadas. El sistema no solo registra compras y devoluciones, sino que también incorpora una característica distintiva: la gestión de préstamos. Esta funcionalidad es particularmente útil en localidades pequeñas, donde las relaciones de confianza entre comerciantes y clientes permiten prácticas como llevarse productos sin un desembolso inicial. Los clientes pueden probar los artículos antes de comprometerse a la compra o optar por la devolución sin costos adicionales.\\


El software proporcionará una interfaz de usuario intuitiva compuesta de múltiples pantallas, entre las que podemos destacar un inventario actualizado en tiempo real, la visualización de los artículos en venta o una lista de clientes habituales del negocio. Además, para impulsar la toma de buenas decisiones estratégicas, el programa generará gráficos analíticos que reflejarán el progreso económico del negocio. \\

Este proyecto surge de la necesidad evidente de una solución tecnológica adaptada a los requisitos específicos de los negocios minoristas. Hasta el momento, son negocios que no se han tenido en cuenta debido a su bajo impacto y su pequeño tamaño. Sin embargo, debemos de mirar por llevar la tecnología a todos los ámbitos. Esta aplicación busca llenar ese vacío y convertirse en una herramienta útil para comerciantes minoristas.  


\cleardoublepage


%\thispagestyle{empty}


\begin{center}
{\large\bfseries Application for the management of a retail store}\\
\end{center}
\begin{center}
Julia María Cano Flores\\
\end{center}

\vspace{0.7cm}
\noindent{\textbf{Keywords}: retailer, database, software, requirements}\\

\vspace{0.7cm}
\noindent{\textbf{Abstract}}\\


The main purpose of the development of this Final Degree Project (FDP) is to simplify and improve the efficiency of retail business management. It is designed to optimize the sales tracking process through a database that records all transactions made. The system not only records purchases and returns but also incorporates a distinctive feature: loan management. This functionality is particularly useful in small localities, where trust relationships between merchants and customers allow practices such as taking products without an initial outlay. Customers can test items before committing to purchase or opt for a return without additional costs.\\

The software will provide an intuitive user interface composed of multiple screens, among which we can highlight an inventory updated in real-time, the display of items for sale, or a list of regular customers of the business. In addition, to drive good strategic decision-making, the program will generate analytical charts that will reflect the economic progress of the business.\\

This project arises from the clear need for a technological solution adapted to the specific requirements of retail businesses. So far, they are businesses that have not been considered due to their low impact and small size. However, we must look to bring technology to all areas. This application seeks to fill that void and become a useful tool for retail merchants.



\chapter*{}

%\thispagestyle{empty}


\noindent\rule[-1ex]{\textwidth}{2pt}\\[4.5ex]

Yo, \textbf{Julia María Cano Flores}, alumna de la titulación INGENIERÍA INFORMÁTICA de la \textbf{Escuela Técnica Superior
de Ingenierías Informática y de Telecomunicación de la Universidad de Granada}, con DNI 77649643x, autorizo la
ubicación de la siguiente copia de mi Trabajo Fin de Grado en la biblioteca del centro para que pueda ser
consultada por las personas que lo deseen.

\vspace{6cm}

\noindent Fdo: Julia María Cano Flores

\vspace{2cm}

\begin{flushright}
Granada a 2 de febrero de 2024 .
\end{flushright}


\chapter*{}
%\thispagestyle{empty}

\noindent\rule[-1ex]{\textwidth}{2pt}\\[4.5ex]

D. \textbf{María Luisa Rodríguez Almendros}, Profesora del Área de Ingeniería del Software del Departamento de Lenguajes y Sistemas Informáticos de la Universidad de Granada.




\vspace{0.5cm}

\textbf{Informan:}

\vspace{0.5cm}

Que el presente trabajo, titulado \textit{\textbf{Aplicación para la gestión de una tienda minorista}},
ha sido realizado bajo su supervisión por \textbf{Julia María Cano Flores}, y autorizamos la defensa de dicho trabajo ante el tribunal
que corresponda.

\vspace{0.5cm}

Y para que conste, expiden y firman el presente informe en Granada a 2 de febrero de 2024 .

\vspace{1cm}

\textbf{Los directores:}

\vspace{5cm}

\noindent \textbf{María Luisa Rodríguez Almendros}

\chapter*{Agradecimientos}
\thispagestyle{empty}

       \vspace{1cm}


Poner aquí agradecimientos...




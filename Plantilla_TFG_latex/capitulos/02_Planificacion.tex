\chapter{Planificación}
\label{chap:planification}


\section{Fases}

La planificación del desarrollo del proyecto se ha dividido en las fases que vamos a ver a continuación.

\subsection{Fase inicial}

En esta etapa es donde se definen las bases del proyecto. Se identifican las necesidades y se establecen unos objetivos principales que deberán de cumplirse a lo largo del desarrollo del proyecto. Esta información podemos verla en el capítulo 1 (\hyperref[chap:introduction]{ver página}) de este documento. Además, se realiza un estudio de la viabilidad y un caso de negocio, donde se pueda observar los beneficios que la aplicación es capaz de producir frente a los gastos y los riesgos. 

\subsection{Fase de análisis}

Durante la fase de análisis se estudia de forma minuciosa los requisitos del proyecto. Se deben comprender de forma exacta las necesidades del cliente y posteriormente convertirlas en una lista de especificaciones que el software debe cumplir. \\ 

Además se realiza un análisis de los posibles riesgos que nos podemos encontrar en el desarrollo del proyecto. Con todo esto, se elabora un plan que englobe todos los requisitos y tenga en cuenta las amenazas del proyecto y cómo actuar ante ellas. Tras esta fase de análisis se debe tener una planificación buena para conseguir el desarrollo exitoso del proyecto. 


\subsection{Fase de desarrollo}

Esta es la fase donde se construye el producto que hemos definido en la fase de análisis. Suele ser la etapa más larga ya que se transforman todos los planes y diseños previos en código, dando como resultado un entregable final funcional.\\

Durante la fase de desarrollo suelen haber principalmente tareas de diseño, construcción, codificación, integración y prototipado. Con la realización de cada una de estas tareas se consigue un producto final al terminar la fase de desarrollo. 

\subsection{Fase de prueba y corrección}

Tras terminar la fase de desarrollo es importante comprobar que aquello construido hace sus funcionalidades de forma correcta. Para ello se inicia una fase de prueba y corrección, donde el producto se evalúa y se comprueban que los resultados obtenidos sean los esperados. 

\subsection{Redacción de la memoria}

Con el objetivo de documentar el desarrollo del proyecto, se va a plasmar toda la información relevante en este documento. 

Para conseguir un buen diseño y ordenación del documento se va utilizar la herramienta de LaTeX. 

\newpage

\section{Presupuesto}

\subsection{Recursos}

En este apartado se va a exponer los recursos hardware y software que se van a utilizar para el desarrollo del proyecto. 

\begin{adjustwidth}{2em}{0pt} % Aumenta el margen izquierdo por 2em
\subsubsection{Hardware}

Los componentes hardware que se van a utilizar para llevar a cabo el proyecto son: 

\begin{itemize}
	\item \textbf{Ordenador:} MSI Prestige 15 A10SC.
\end{itemize}

\subsubsection{Software}

Las herramientas software que se van a utilizar para llevar a cabo el proyecto son: 

\begin{itemize}
	\item \textbf{Sistema Operativo:} Ubuntu 20.04
	\item \textbf{Lenguaje de programación:} Flutter
	\item \textbf{IDE:} Visual Studio Code
	\item \textbf{Diseño de diagramas UML:} Visual Paradigm
	\item \textbf{Sistema de composición de texto:} LaTeX
	\item \textbf{Editor de texto:} TeXstudio
\end{itemize}
\end{adjustwidth} 

\subsection{Costes}

En esta sección vamos a analizar el coste total del proyecto. Vamos a distinguir distintos apartados. 

\begin{adjustwidth}{2em}{0pt} % Aumenta el margen izquierdo por 2em
	\subsubsection{Licencias}
	
	Un tipo de coste que conlleva el desarrollo de un proyecto es la adquisición de las licencias necesarias para la producción del mismo. Las licencias que se van a utilizar en este proyecto son de software libre y por tanto son gratuitas. Esto significa que no supondrán ningún coste adicional. Las licencias que se utilizarán son las siguientes: 

\newpage

	\begin{itemize}
		\item \textbf{Ubuntu 20.04:} GNU General Public Licence (GPL).
		\item \textbf{Flutter:} Licencia BSD. 
		\item \textbf{Visual Paradigm:} Licencia adquirida por usos académicos. 
		\item \textbf{LaTeX:} LaTeX Project Public License (LPPL).
	\end{itemize}
	
	
	\subsubsection{Recursos materiales}
	
	El único recurso material que se va utilizar para el desarrollo del proyecto es el ordenador personal. \\
	
	El periodo de amortización común para los ordenadores y equipos informáticos es de 3 a 5 años. Realizaremos la media y utilizaremos un periodo de 4 años para los cálculos. Sabiendo que el equipo costó 1400€, se amortizará 350€ al año. Como la duración del proyecto es de 5 meses, el coste final de los recursos materiales será de 145'83€
	
	\subsubsection{Recursos humanos}
	
	En esta sección se incluyen los gastos por la contratación de personal. Este proyecto solamente lo va a desarrollar una persona, bajo la titulación de programador senior. \\
	
	En la actualidad, un programador senior recibe una media de 22.000€ anuales. Durante los 5 meses que dura el proyecto, se estima un salario de 9166'66€. 
	
	\subsubsection{Otros}
	
	Este apartado engloba costes indirectos como los gastos debidos a la localización para trabajar, los gastos de transporte, conexión a Internet, etc. Este gasto se suele aproximar a un 10\% de los gastos de recursos humanos. Por tanto, la cantidad estipulada para este apartado sería de 916'66€. 
	
	\newpage
	
	\subsubsection{Total}
	
% Tabla presupuesto total
\begin{table}[htb!]
	\centering % Centra la tabla en la página
	\begin{tabular}{|p{0.3\linewidth}|p{0.35\linewidth}|p{0.35\linewidth}|}
		\hline
		\rowcolor{grayshade} \textbf{Descripción} & \textbf{Coste mensual} & \textbf{Coste total} \\
		\hline
		\textbf{Licencias} & 0€ & 0€ \\
		\hline
		\textbf{Recursos materiales} & 29'17€ & 145'83€ \\
		\hline
		\textbf{Recursos humanos} & 1833'33€ & 9166'66€ \\
		\hline
		\textbf{Otros} & 183'33€ & 916'66€ \\
		\hline
		\textbf{Total} & 2045,83€ & 10.229'15€ \\
		\hline
	\end{tabular}
	\caption{Presupuesto total}
\end{table}
\end{adjustwidth} 

\chapter{Legislación actual de protección de datos}
\label{chap:law}

La ley de protección de datos existe para proteger y controlar los datos personales de los usuarios, un derecho fundamental que tienen todas las personas. Esta ley evita que nuestros datos personales sean usados para vulnerar nuestra intimidad u otros derechos fundamentales o libertades. En España esta ley lleva vigente desde el año 2000. \\

Dentro de los datos personales hay distintos niveles de seguridad: bajo, medio y alto. Cada uno de estos niveles tiene distintas medidas de seguridad. Las principales medidas de la ley de protección de datos en España son: 

\begin{itemize}
	\item Solo se deben recoger aquellos datos necesarios para el funcionamiento de la aplicación.  
	\item Se debe avisar a la persona cuáles son los datos que van a ser recogidos y con qué fin van a usarse. 
	\item La persona tiene derecho a dar y deshacer el consentimiento de esos datos si existe una causa justificada. 
	\item Los datos de medio y alto nivel no deben ser recogidos a menos que sea estrictamente necesario. Información como la religión o ideología son datos que no deberían de pedirse. 
	\item La persona que interactúe con datos personales protegidos bajo esta ley deberá de cumplir el secreto profesional. 
	\item Se deben tomar medidas técnicas y organizativas para garantizar la seguridad de los datos en todo momento. 
\end{itemize} 

En nuestro caso concreto, los datos que vamos a pedir a los clientes son únicamente tres campos: el nombre, los apellidos y un número de teléfono, este último de carácter opcional. Son los datos estrictamente necesarios para identificar un cliente. No se pedirán datos innecesarios, como bien postula la ley de protección de datos. 

Puesto que los clientes pueden tomar prestados artículos de la tienda, será necesario un número de teléfono para llamar en caso de que se olviden de devolver o comprar dichos artículos. Si el cliente no quiere dar el número de teléfono o no dispone de uno de ellos, bastará con el nombre y los apellidos. 

Estos datos se almacenarán en una base de datos protegida de manera que se pueda garantizar la seguridad de dichos datos. 

\chapter{Análisis de tecnologías a utilizar}
\label{chap:Technologies}


\section{Dispositivo destino para la aplicación}

La aplicación está pensada para ser ejecutada en un dispositivo tipo \textbf{tablet}. De esta forma conseguimos una pantalla lo suficientemente grande como para visualizar y utilizar la aplicación cómodamente. Si fuera necesario, una tablet también permite conectar un teclado para introducir datos más rápidamente o simular el funcionamiento de un ordenador. Además es un dispositivo altamente portable, lo que permite usar la aplicación en cualquier lugar. Por todos estos motivos, se trata del dispositivo más apropiado. 

\section{Tecnología de documentación}

Para documentar se va a utilizar la herramienta de \textbf{LaTeX} ya que es bastante útil a la hora de mantener un documento bien estructurado y con el mismo formato. Además, es una herramienta que se suele utilizar para realizar documentos técnicos y es interesante aprenderla para futuros proyectos. 

\subsection{Descripción de LaTeX}

LaTeX es un sistema de composición de textos de alta calidad, muy utilizado para la creación de documentos científicos y técnicos debido a su gran capacidad para manejar fórmulas matemáticas, referencias cruzadas y bibliografías, entre otras. Permite establecer un buen formato y dar una buena estructura al documento de forma sencilla. \cite{latex}

Algunas de sus principales características es: 

\begin{itemize}
	\item \textbf{Documentos estructurados:} Permite estructurar documentos de manera lógica con capítulos, secciones y subsecciones, lo que resulta ideal para documentos largos y complejos.
	\item \textbf{Gestión de referencias y citas:} Integra sistemas para la gestión automática de citas y bibliografías como BibTeX y Biber.
	\item \textbf{Sistema de tipografía:} LaTeX permite la tipografía de calidad editorial con características profesionales.
	\item \textbf{Licencia de software libre:} Es un software de código abierto, lo que significa que es gratuito y se puede modificar y distribuir.
\end{itemize}  

\begin{figure}[H]
	\centering
	\includegraphics[width=0.5\textwidth]{imagenes/imagenesTecnologias/latex.png}
	\caption{Logo LaTeX}
\end{figure}

\section{Tecnología para la realización de los diagramas de casos de uso}

Para realizar los diagramas de casos de uso se ha utilizado \textbf{Visual Paradigm}. He escogido esta herramienta ya que hemos trabajado anteriormente con ella, conozco su funcionamiento y es muy útil. Se requiere de una licencia para utilizarla, pero la UGR la proporciona gratuitamente a sus estudiantes. 

\subsection{Descripción de Visual Paradigm}

Visual Paradigm es una herramienta de modelado de software. La plataforma ofrece modelado de negocio completo, soporte para una amplia gama de diagramas, automatización de flujos de trabajo, y colaboración en equipo. Incluye un repositorio en la nube y facilita la creación de prototipos, maquetas y wireframes. \cite{visual}

\newpage

Algunos de los diagramas más destacados que ofrece son: 

\begin{itemize}
	\item Diagramas de clase UML. 
	\item Diagramas de casos de uso UML. 
	\item Diagramas de secuencia UML. 
	\item Diagramas de Entidad-Relación (ER).
	\item Diagramas de arquitectura de software. 
\end{itemize}  

\begin{figure}[H]
	\centering
	\includegraphics[width=0.7\textwidth]{imagenes/imagenesTecnologias/Visual_Paradigm_logo.png}
	\caption{Logo Visual Paradigm}
\end{figure}


\section{Tecnología para el diseño de los bocetos}

Para el diseño de los bocetos he utilizado \textbf{Goodnotes}, una aplicación para escribir en el Ipad. Me ha parecido la forma más sencilla y rápida de hacer unos bocetos. Además, al ser dibujado, hay total libertad para expresar el diseño tal y como se desea hacer. Es un diseño minimalista, pero se puede observar de una manera muy visual cuál será la interfaz de usuario de la aplicación a desarrollar. 

\subsection{Descripción de Goodnotes}

Goodnotes es una aplicación de toma de notas digital que se destaca por combinar tecnología de inteligencia artificial con funcionalidades tradicionales para crear una experiencia de usuario avanzada y versátil. \cite{goodnotes}

\begin{itemize}
	\item \textbf{Herramientas de escritura y dibujo flexibles:}  Permite combinar texto escrito a mano y tecleado en una sola página.
	\item \textbf{Búsqueda potente:} Permite buscar y encontrar rápidamente texto escrito a mano.
	\item \textbf{Personalización y organización: } Ofrece opciones para personalizar plantillas.
	\item \textbf{Gestos intuitivos con el lápiz: } Incluye gestos como "Scribble to Erase" para borrar rápidamente.
\end{itemize}

\begin{figure}[H]
	\centering
	\includegraphics[width=0.7\textwidth]{imagenes/imagenesTecnologias/goodnotes.jpg}
	\caption{Logo GoodNotes}
\end{figure}


\section{Metodología de desarrollo de la aplicación}

Se utilizará una \textbf{metodología iterativa}. El desarrollo de la aplicación se dividirá en varias iteraciones, cada una con sus requisitos correspondientes y su entregable final. De esta forma, se realizará una revisión constante de aquello que se está construyendo con el objetivo de no desviarnos de las necesidades del usuario y consiguiendo una mayor adaptación a lo largo del proyecto. 

\subsection{Descripción de metodología iterativa}

Una metodología iterativa es un enfoque de desarrollo de proyectos en ciclos repetitivos, permitiendo la adaptación y mejora continua basada en el feedback después de cada iteración. Es especialmente útil en el desarrollo de software para responder a cambios y resolver problemas rápidamente. \cite{agile}

Las principales características son: 

\begin{itemize}
	\item \textbf{Iteraciones: } El desarrollo se divide en iteraciones, cada una resultando en una versión mejorada del producto, permitiendo ajustes regulares según sea necesario.
	\item \textbf{Feedback continuo: } Al final de cada iteración se evalúa el progreso y se recopila feedback para guiar el desarrollo futuro.
	\item \textbf{Flexibilidad en la planificación: } Permite ajustes en el plan basados en el aprendizaje del proyecto y los cambios de requisitos. 
	\item \textbf{Entregables frecuentes: } Proporciona versiones funcionales del producto en etapas tempranas, lo que ayuda a verificar la viabilidad y la dirección del proyecto. 
	\item \textbf{Mejora continua: } La naturaleza iterativa fomenta la revisión y mejora constante del producto. 
\end{itemize}

\begin{figure}[H]
	\centering
	\includegraphics[width=0.7\textwidth]{imagenes/imagenesTecnologias/iterativa.jpg}
	\caption{Metodología iterativa}
\end{figure}

\section{Tecnología de soporte de la base de datos}

La tecnología utilizada para la creación y utilización de la base de datos del proyecto es \textbf{Supabase}. He escogido esta alternativa ya que proporciona una opción de base de datos en línea, que evita tener que instalar un propio servidor. Además, las bases de datos son relacionales lo que permite una mejor ordenación y acceso de la información. 

\subsection{Descripción de Supabase}

Supabase es una plataforma de desarrollo backend como servicio diseñada para ayudar a los desarrolladores a construir aplicaciones rápidamente. Es una alternativa de código abierto a Firebase. Sus principales características por las que destaca son: \cite{supabase}

\begin{itemize}
	\item \textbf{Base de datos PostgreSQL: } Es una base de datos relacional, robusta y escalable. 
	\item \textbf{API RESTful generada automáticamente: } Se genera automáticamente una API basada en la estructura de la base de datos. Esto permite interactuar con la información de forma sencilla. 
	\item \textbf{Autenticación y autorización: } Incluye un sistema de autenticación completo que permite iniciar sesión de múltiples formas. Además, proporciona un control de acceso a las tablas para garantizar la seguridad. 
	\item \textbf{Subscripciones en tiempo real: } Permite agregar funcionalidades en tiempo real a sus aplicaciones a través de subscripciones a eventos. 
	\item \textbf{Integraciones y extensibilidad: } Se integra fácilmente con otras herramientas y servicios populares. 
\end{itemize}

\begin{figure}[H]
	\centering
	\includegraphics[width=0.7\textwidth]{imagenes/imagenesTecnologias/supabase.jpg}
	\caption{Logo Supabase}
\end{figure}

\section{Framework de desarrollo}

El framework de desarrollo que he escogido para implementar la aplicación ha sido \textbf{Flutter}. El principal motivo por el que he escogido esta teconología es por ser multiplataforma, lo que permite transladar la aplicación a android, ios o web. Esto amplia la compatibilidad de la app con diferentes dispositivos. Además, tiene multitud de widgets que te permiten desarrollar una aplicación de forma más eficiente. El lenguaje de programación utilizado es \textbf{Dart}. 


\subsection{Descripción de Flutter}

Flutter permite construir aplicaciones nativas de alta calidad para múltiples plataformas utilizando una única base de código. \textbf{Dart} es el lenguaje de programación utilizado por Flutter. Es un lenguaje sencillo de aprender y usar. Las principales características de Flutter son:  \cite{flutter}

\begin{itemize}
	\item \textbf{UI Nativa y rápida: } Flutter utiliza su propio motor de renderizado para dibujar widgets directamente en la pantalla. Con esto se logra una sensación nativa en cada plataforma. 
	\item \textbf{Desarrollo con un solo código base: } El código se escribe en Dart y se adapta a las distintas plataformas. 
	\item \textbf{Hot reload: } Permite ver los cambios en el código de inmediato, sin tener que reiniciar la ejecución. 
	\item \textbf{Extensa biblioteca de widgets: } Ofrece una amplia variedad de widgets personalizables para construir interfaces de usuario complejas y atractivas. 
	\item \textbf{Alto rendimiento: } Flutter ofrece un rendimiento cercano al nativo. 
\end{itemize}

\begin{figure}[H]
	\centering
	\includegraphics[width=0.5\textwidth]{imagenes/imagenesTecnologias/flutter.jpg}
	\caption{Logo Flutter}
\end{figure}



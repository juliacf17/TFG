\chapter{Análisis de tecnologías a utilizar}
\label{chap:Technologies}

\section{Tecnología de documentación}

Para documentar se va a utilizar la herramienta de \textbf{LaTeX} ya que es bastante útil a la hora de mantener un documento bien estructurado y con el mismo formato. Además, es una herramienta que se suele utilizar para realizar documentos técnicos y es interesante aprenderla para futuros proyectos.  

\section{Dispositivo destino para la aplicación}

La aplicación está pensada para ser ejecutada en un dispositivo tipo \textbf{tablet}. De esta forma conseguimos una pantalla lo suficientemente grande como para visualizar y utilizar la aplicación cómodamente. Si fuera necesario, una tablet también permite conectar un teclado para introducir datos más rápidamente o simular el funcionamiento de un ordenador. Además es un dispositivo altamente portable, lo que permite usar la aplicación en cualquier lugar. Por todos estos motivos, se trata del dispositivo más apropiado. 

\section{Tecnología para la realización de los diagramas de casos de uso}

Para realizar los diagramas de casos de uso se ha utilizado \textbf{Visual Paradigm}. He escogido esta herramienta ya que hemos trabajado anteriormente con ella, conozco su funcionamiento y es muy útil. Se requiere de una licencia para utilizarla, pero la UGR la proporciona gratuitamente a sus estudiantes. 



\section{Tecnología para el diseño de los bocetos}

Para el diseño de los bocetos he utilizado \textbf{Goodnotes}, una aplicación para escribir en el Ipad. Me ha parecido la forma más sencilla y rápida de hacer unos bocetos. Además, al ser dibujado, hay total libertad para expresar el diseño tal y como se desea hacer. Es un diseño minimalista, pero se puede observar de una manera muy visual cuál será la interfaz de usuario de la aplicación a desarrollar. 


\section{Metodología de desarrollo de la aplicación}

Se utilizará una \textbf{metodología iterativa}. El desarrollo de la aplicación se dividirá en varias iteraciones, cada una con sus requisitos correspondientes y su entregable final. De esta forma, se realizará una revisión constante de aquello que se está construyendo con el objetivo de no desviarnos de las necesidades del usuario y consiguiendo una mayor adaptación a lo largo del proyecto. 
\chapter{Análisis de los negocios minoristas}
\label{chap:StoreAnalysis}

Un negocio minorista es una empresa que vende productos o servicios directamente a consumidores finales. Los minoristas actúan como intermediarios entre los fabricantes, mayoristas o distribuidores y el mercado de consumo. Por tanto, es responsabilidad del minorista conocer cuál es la cantidad de producto que debe comprar para asegurarse de vender lo máximo posible y conseguir con ello el mayor beneficio. \\

Las tiendas minoristas pueden ser de múltiples tamaños, ya que no son tiendas minoristas únicamente las pequeñas, sino que serán todas aquellas que ofrezcan productos a usuarios finales. Sin embargo, en este proyecto nos centraremos en la gestión de las tiendas minoristas pequeñas, en las microempresas minoristas. Una microempresa se caracteriza por tener una pequeña cantidad de empleados, menos de 10, y un volumen de ventas reducido. \\

Las microempresas minoristas suelen ser negocios familiares, como tiendas de barrio, puestos de mercado o tiendas especializadas que operan a una escala local. Estas empresas suelen ofrecer una gama de productos limitada. Además, estos negocios suelen tener un trato con el cliente más personalizado, lo que les permite diferenciarse de las empresas más grandes. \\

Este proyecto en concreto va destinado al negocio minorista de mi padre. Es una tienda de una pequeña pedanía de Lorca en la que ofrece ropa y calzado para todas las edades, complementos, material de costura y textiles para el hogar. Es una tienda sin empleados, donde el único responsable del funcionamiento de la tienda es mi padre.  



\chapter{Análisis de usabilidad y accesibilidad}
\label{chap:Accessibility}

Debemos buscar la usabilidad y accesibilidad en una aplicación de estas características ya que el usuario final puede no tener grandes destrezas tecnológicas. Por tanto, hay que conseguir un diseño sencillo, fácil de usar e intuitivo. \\

Para ello, habrá que tener en cuenta los siguientes criterios:

\begin{itemize}
	\item \textbf{Botones grandes:} Los botones de la aplicación deben de tener un tamaño adecuado para facilitar la realización de acciones. 
	\item \textbf{Tamaño y contraste del texto:} Nos debemos de asegurar que el tamaño de la letra es lo suficientemente grande y que el contraste con el fondo es significativo para facilitar con ello la lectura. El fondo de la aplicación será blanco para conseguir un buen contraste y evitar las distracciones. 
	\item \textbf{Claridad en la navegación:} Proporcionar una estructura lógica de navegación y unas flechas visibles que permitan cambiar de pantallas de manera intuitiva. Además, se deberá añadir una barra superior que indicará la pantalla en la que estamos situados para conseguir mayor claridad. 
	\item \textbf{Simplicidad:} El diseño debe ser sencillo. Evitar la sobrecarga de información y priorizar un diseño ordenado. 
	\item \textbf{Feedback:} La aplicación debe proporcionar una respuesta a las acciones del usuario para confirmar que se realizan de manera correcta. 
\end{itemize} 

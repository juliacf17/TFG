\chapter{Introducción}
\label{chap:introduction}

\section{Motivación}
\label{sec:motivation}

Se entiende como negocio minorista toda empresa de comercio que adquiere mercancías por cuenta propia, y las revende directamente al consumidor final. En negocios minoristas pequeños podemos encontrar un trato más cercano con los clientes y una forma algo distinta de gestionar un comercio. \\

Al ser nacida y criada en un pueblo, la mayoría de negocios con los que he crecido eran comercios minoristas pequeños. Además, mi padre es propietario de una tienda y he podido conocer de primera mano cuál es el funcionamiento real de este tipo de negocios. Tras una vida tratando con ellos y una entrevista realizada a mi padre, he encontrado la necesidad de informatizar este trabajo con el objetivo de conseguir optimizar el rendimiento y llevar un seguimiento. En la actualidad, muchos comerciantes minoristas no disponen de grandes tecnologías y recurren a métodos tradicionales basados en papel para el seguimiento de su negocio. Esto tiene un riesgo alto de pérdida de datos o cometer errores. 
Además, el proceso de recuperación y análisis de datos se convierten en tareas muy tediosas, ya que la falta de sistemas automatizados impide un filtrado eficiente de información. Todo esto ralentiza el trabajo de forma significativa, haciendo que la productividad del comerciante sea menor y, por tanto, la evolución del negocio no sea óptima. \\

En el mercado actual no encontramos aplicaciones que pongan solución al problema de los negocios minoristas pequeños. Las soluciones software existentes tienden a enfocarse en entidades de mayor escala. Como hemos visto, los negocios minoristas tienen singularidades que conllevan establecer requisitos específicos y tratar su gestión de forma independiente a la gestión de un gran negocio. Por ello, es crucial el desarrollo de un software que satisfaga las necesidades de los pequeños comerciantes y les permita conducir su negocio a una evolución óptima. 


\section{Objetivos del proyecto}
\label{sec:project_objectives}

El objetivo general de este proyecto es proporcionar una solución informática analizando los procedimientos típicos de un comercio minorista pequeño que actualmente se realizan de forma manual y desarrollando una aplicación que los automatice. Para conseguir dicho objetivo, podemos diferenciar una serie de objetivos específicos: 


% Tabla para el Objetivo 1
\definecolor{grayshade}{gray}{0.9}



\begin{itemize}
	\item \textbf{Objetivo 1: Análisis de los comercios minoristas}
	\begin{itemize}
		\item \textbf{Descripción:} Se debe de realizar una búsqueda de información sobre los negocios minoristas con el objetivo de entender mejor cuáles son sus características.
	\end{itemize}
	
	\item \textbf{Objetivo 2: Estudio de aplicaciones similares}
	\begin{itemize}
		\item \textbf{Descripción:} Análisis del mercado actual de aplicaciones similares para identificar funcionalidades innovadoras que implementar en nuestra aplicación.
	\end{itemize}
	
	\item \textbf{Objetivo 3: Análisis de usabilidad y accesibilidad}
	\begin{itemize}
		\item \textbf{Descripción:} Estudiar soluciones accesibles para conseguir una aplicación fácil de usar para la mayoría de usuarios.
	\end{itemize}
	
	\item \textbf{Objetivo 4: Análisis de las tecnologías a utilizar}
	\begin{itemize}
		\item \textbf{Descripción:} Investigar sobre cuáles son las mejores tecnologías para llevar a cabo el proyecto.
	\end{itemize}
	
	\item \textbf{Objetivo 5: Desarrollo e implementación de la aplicación}
	\begin{itemize}
		\item \textbf{Descripción:} Analizar, desarrollar y probar la aplicación con el objetivo de conseguir los mejores resultados.
	\end{itemize}
	
	\item \textbf{Objetivo 6: Validación de la aplicación por usuarios reales}
	\begin{itemize}
		\item \textbf{Descripción:} Poner a prueba la aplicación en un entorno real para ver si cumple las expectativas del usuario.
	\end{itemize}
\end{itemize}

\section{Estructura del documento}
\label{sec:document_structure}
Esta sección se ofrece una visión de la estructura y organización del documento. Se compone de las siguientes partes: 

\begin{itemize}
	\item \textbf{Resumen:} Un resumen donde se exponen las principales funcionalidades del proyecto y sus objetivos. 
	
	\item \textbf{Introducción:} En este apartado se exponen los principales objetivos que se pretenden cumplir durante el desarrollo del proyecto. Además, se explica la motivación por la que se empezó a idear dicho proyecto a modo de introducción. 

	
	\item \textbf{Planificación:} Especificación de la planificación del proyecto, donde se muestran las distintas fases y la estimación presupuestaria del mismo. 
	
	\item \textbf{Estado del arte:} Investigación sobre los principales temas que se tratan en el proyecto como los negocios minoristas, los negocios mixtos y los comerciantes minoristas. Además, se tratan temas legales como la protección de datos. Con esta investigación, entramos en contexto para posteriormente entender los requisitos del proyecto. 
	
	\item \textbf{Análisis de tecnologías a utilizar:} Enumeración de tecnologías utilizadas para el desarrollo del proyecto. Para cada tecnología, se explica porqué se ha utilizado y se citan sus principales características. 
	
	\item \textbf{Análisis de la aplicación:} Obtención y explicación de los requisitos del sistema. Posteriormente se amplia esta información desarrollando los casos de uso, el modelo conceptual y los diagramas de secuencia del sistema. 
	
	\item \textbf{Diseño:} Descripción de la arquitectura del sistema, el diseño de las clases necesarias para el programa y la interfaz de usuario.
	
	\item \textbf{Implementación de la aplicación:} La implementación de la aplicación se desarrollará en con una metodología ágil, por lo que este capítulo se dividirá en sus iteraciones. 
	
	\item \textbf{Bibliografía:} Recopilación de todas las fuentes de información utilizadas durante la realización del proyecto.
\end{itemize}
\newpage






\chapter{Introducción}
\label{chap:introduction}

\section{Motivación}
\label{sec:motivation}
Un negocio minorista es aquel que vende pequeñas cantidades a clientes finales en una tienda física normalmente situada en entornos rurales. En este tipo de negocios podemos encontrar un trato más cercano con los clientes y una forma algo distinta de gestionar un negocio. \\

Tras investigar los negocios minoristas, he encontrado la necesidad de informatizar este trabajo con el objetivo de conseguir optimizar el rendimiento y llevar un seguimiento. En la actualidad, muchos comerciantes minoristas no disponen de grandes tecnologías y recurren a métodos tradicionales basados en papel para el seguimiento de su negocio. Esto tiene un riesgo alto de pérdida de datos o cometer errores. 
Además, el proceso de recuperación y análisis de datos se convierten en tareas muy tediosas, ya que la falta de sistemas automatizados impide un filtrado eficiente de información. Todo esto ralentiza el trabajo de forma significativa, haciendo que la productividad del comerciante sea menor y, por tanto, la evolución del negocio no sea óptima. \\

En el mercado actual no encontramos aplicaciones que pongan solución al problema de los negocios minoristas. Las soluciones software existentes tienden a enfocarse en entidades de mayor escala. Como hemos visto, los negocios minoristas tienen singularidades que conllevan establecer requisitos específicos y tratar su gestión de forma independiente a la gestión de un gran negocio. Por ello, es crucial el desarrollo de un software que satisfaga las necesidades de los pequeños comerciantes y les permita conducir su negocio a una evolución óptima. 

  

\section{Estructura del documento}
\label{sec:document_structure}
Esta sección se ofrece una visión de la estructura y organización del documento. Se compone de las siguientes partes: 

\begin{itemize}
	\item \textbf{Resumen:} Un resumen donde se exponen las principales funcionalidades del proyecto y sus objetivos. 
	
	\item \textbf{Introducción:} Encontramos tres secciones iniciales: 
	\begin{itemize}
		\item La motivación del proyecto.
		\item La estructura general del documento.
		\item Los objetivos que se pretenden cumplir durante el desarrollo del proyecto.
	\end{itemize}
	
	\item \textbf{Planificación:} Especificación de la planificación del proyecto, donde se muestran las fases y la estimación presupuestaria del mismo. 
	
	\item \textbf{Análisis:} Especificación de los requisitos del sistema y los casos de uso. 
	
	\item \textbf{Diseño:} Descripción de la estructura y el diseño de las clases necesarias para el programa y la interfaz de usuario.
	
	\item \textbf{Implementación:} Aspectos a destacar de la implementación del proyecto, problemas encontrados y soluciones aplicadas. Podemos distinguir tres subsecciones: 
	\begin{itemize}
		\item Herramientas utilizadas y justificación de su elección.
		\item Implementaciones clave durante las fases de desarrollo.
		\item Diseño de la interfaz de usuario.
	\end{itemize}
	
	\item \textbf{Conclusiones:} Resumen de los resultados obtenidos y contemplación de posibles direcciones futuras para el proyecto.
	
	\item \textbf{Bibliografía:} Recopilación de todas las fuentes de información utilizadas durante la realización del proyecto.
	
	\item \textbf{Anexos:} Incluye documentos adicionales de valor para el proyecto, como un glosario de términos y acrónimos relevantes.
\end{itemize}
\newpage

\section{Objetivos del proyecto}
\label{sec:project_objectives}
En este apartado vamos a exponer los principales objetivos del proyecto. Podemos diferenciar dos tipos de objetivos en función de su importancia dentro del proyecto: objetivos obligatorios y objetivos opcionales. Los objetivos obligatorios son aquellos que definen la esencia del proyecto mientras que los objetivos opcionales son los que aportan valor adicional al proyecto. A continuación vamos a listar los objetivos: 

% Tabla para el Objetivo 1
\definecolor{grayshade}{gray}{0.9}


% Tabla para el Objetivo 1
\begin{table}[htb!]
	\centering % Centra la tabla en la página
	\begin{tabular}{|p{0.3\linewidth}|p{0.6\linewidth}|}
		\hline
		\rowcolor{grayshade} \textbf{Objetivo 1} & \textbf{Seguimiento de ventas, devoluciones y préstamos} \\
		\hline
		\textbf{Tipo} & Obligatorio \\
		\hline
		\textbf{Descripción} & El programa debe ser capaz almacenar y gestionar todos los movimientos que se realizan en el negocio. \\
		\hline
	\end{tabular}
	\caption{Objetivo 1}
\end{table}

% Tabla para el Objetivo 2
\begin{table}[htb!]
	\centering % Centra la tabla en la página
	\begin{tabular}{|p{0.3\linewidth}|p{0.6\linewidth}|}
		\hline % Asegúrate de que esta línea esté aquí
		\rowcolor{grayshade} 
		\textbf{Objetivo 2} & \textbf{Creación de un inventario} \\
		\hline % Duplicar esta línea después de \rowcolor puede ayudar
		\textbf{Tipo} & Obligatorio \\
		\hline
		\textbf{Descripción} & El programa debe ser capaz de mostrar un inventario actualizado en tiempo real con los artículos disponibles del negocio. \\
		\hline
	\end{tabular}
	\caption{Objetivo 2}
\end{table}

% Tabla para el Objetivo 3
\begin{table}[htb!]
	\centering % Centra la tabla en la página
	\begin{tabular}{|p{0.3\linewidth}|p{0.6\linewidth}|}
		\hline % Asegúrate de que esta línea esté aquí
		\rowcolor{grayshade} 
		\textbf{Objetivo 3} & \textbf{Creación y administración de perfiles de clientes habituales} \\
		\hline % Duplicar esta línea después de \rowcolor puede ayudar
		\textbf{Tipo} & Obligatorio \\
		\hline
		\textbf{Descripción} & El programa debe ser capaz de registrar nuevos clientes del negocio y gestionar los movimientos que realizan dichos clientes (compras, devoluciones o préstamos). \\
		\hline
	\end{tabular}
	\caption{Objetivo 3}
\end{table}

% Tabla para el Objetivo 4
\begin{table}[htb!]
	\centering % Centra la tabla en la página
	\begin{tabular}{|p{0.3\linewidth}|p{0.6\linewidth}|}
		\hline % Asegúrate de que esta línea esté aquí
		\rowcolor{grayshade} 
		\textbf{Objetivo 4} & \textbf{Creación y gestión de los artículos} \\
		\hline % Duplicar esta línea después de \rowcolor puede ayudar
		\textbf{Tipo} & Opcional \\
		\hline
		\textbf{Descripción} & El programa debe ser capaz de crear nuevos artículos para luego poder cuantificarlos en el inventario y citarlos en los movimientos del negocio.  \\
		\hline
	\end{tabular}
	\caption{Objetivo 4}
\end{table}

% Tabla para el Objetivo 5
\begin{table}[H]
	\centering % Centra la tabla en la página
	\begin{tabular}{|p{0.3\linewidth}|p{0.6\linewidth}|}
		\hline % Asegúrate de que esta línea esté aquí
		\rowcolor{grayshade} 
		\textbf{Objetivo 5} & \textbf{Creación de gráficas basadas en los movimientos del negocio} \\
		\hline % Duplicar esta línea después de \rowcolor puede ayudar
		\textbf{Tipo} & Opcional \\
		\hline
		\textbf{Descripción} & El programa debe ser capaz de analizar los datos recopilados y realizar un resumen en forma de gráfica para que el comerciante pueda ver el progreso económico del negocio.  \\
		\hline
	\end{tabular}
	\caption{Objetivo 5}
\end{table}



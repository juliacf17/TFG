\chapter{Introducción}
\label{chap:introduction}

\section{Motivación}
\label{sec:motivation}

Se entiende como negocio minorista toda empresa de comercio que adquiere mercancías por cuenta propia, y las revende directamente al consumidor final. En negocios minoristas pequeños podemos encontrar un trato más cercano con los clientes y una forma algo distinta de gestionar un comercio. \\

Al ser nacida y criada en un pueblo, la mayoría de negocios con los que he crecido eran comercios minoristas pequeños. Además, mi padre es propietario de una tienda y he podido conocer de primera mano cuál es el funcionamiento real de este tipo de negocios. Tras una vida tratando con ellos y una entrevista realizada a mi padre, he encontrado la necesidad de informatizar este trabajo con el objetivo de conseguir optimizar el rendimiento y llevar un seguimiento. En la actualidad, muchos comerciantes minoristas no disponen de grandes tecnologías y recurren a métodos tradicionales basados en papel para el seguimiento de su negocio. Esto tiene un riesgo alto de pérdida de datos o cometer errores. 
Además, el proceso de recuperación y análisis de datos se convierten en tareas muy tediosas, ya que la falta de sistemas automatizados impide un filtrado eficiente de información. Todo esto ralentiza el trabajo de forma significativa, haciendo que la productividad del comerciante sea menor y, por tanto, la evolución del negocio no sea óptima. \\

En el mercado actual no encontramos aplicaciones que pongan solución al problema de los negocios minoristas pequeños. Las soluciones software existentes tienden a enfocarse en entidades de mayor escala. Como hemos visto, los negocios minoristas tienen singularidades que conllevan establecer requisitos específicos y tratar su gestión de forma independiente a la gestión de un gran negocio. Por ello, es crucial el desarrollo de un software que satisfaga las necesidades de los pequeños comerciantes y les permita conducir su negocio a una evolución óptima. 


\section{Objetivos del proyecto}
\label{sec:project_objectives}

El objetivo general de este proyecto es proporcionar una solución informática a los procedimientos típicos de un comercio minorista pequeño que actualmente se realizan de forma manual. Para conseguir dicho objetivo, podemos diferenciar una serie de objetivos específicos: 


% Tabla para el Objetivo 1
\definecolor{grayshade}{gray}{0.9}


% Tabla para el Objetivo 1
\begin{table}[htb!]
	\centering % Centra la tabla en la página
	\begin{tabular}{|p{0.3\linewidth}|p{0.6\linewidth}|}
		\hline
		\rowcolor{grayshade} \textbf{Objetivo 1} & \textbf{Análisis de los comercios minoristas} \\
		\hline
		\textbf{Descripción} & Se debe de realizar una búsqueda de información sobre los negocios minoristas con el objetivo de entender mejor cuáles son sus características. \\
		\hline
	\end{tabular}
	\caption{Objetivo 1}
\end{table}

% Tabla para el Objetivo 2
\begin{table}[htb!]
	\centering % Centra la tabla en la página
	\begin{tabular}{|p{0.3\linewidth}|p{0.6\linewidth}|}
		\hline % Asegúrate de que esta línea esté aquí
		\rowcolor{grayshade} 
		\textbf{Objetivo 2} & \textbf{Estudio de aplicaciones similares} \\
		\hline 
		\textbf{Descripción} & Análisis del mercado actual de aplicaciones similares para identificar funcionalidades innovadoras que implementar en nuestra aplicación. \\
		\hline
	\end{tabular}
	\caption{Objetivo 2}
\end{table}

% Tabla para el Objetivo 3
\begin{table}[htb!]
	\centering % Centra la tabla en la página
	\begin{tabular}{|p{0.3\linewidth}|p{0.6\linewidth}|}
		\hline % Asegúrate de que esta línea esté aquí
		\rowcolor{grayshade} 
		\textbf{Objetivo 3} & \textbf{Análisis de usabilidad y accesibilidad} \\
		\hline 
		\textbf{Descripción} & Estudiar soluciones accesibles para conseguir una aplicación fácil de usar para la mayoría de usuarios. \\
		\hline
	\end{tabular}
	\caption{Objetivo 3}
\end{table}

% Tabla para el Objetivo 4
\begin{table}[htb!]
	\centering % Centra la tabla en la página
	\begin{tabular}{|p{0.3\linewidth}|p{0.6\linewidth}|}
		\hline % Asegúrate de que esta línea esté aquí
		\rowcolor{grayshade} 
		\textbf{Objetivo 4} & \textbf{Análisis de las tecnologías a utilizar} \\
		\hline % Duplicar esta línea después de \rowcolor puede ayudar
		\textbf{Descripción} & Investigar sobre cuáles son las mejores tecnologías para llevar a cabo el proyecto.  \\
		\hline
	\end{tabular}
	\caption{Objetivo 4}
\end{table}

% Tabla para el Objetivo 5
\begin{table}[htb!]
	\centering % Centra la tabla en la página
	\begin{tabular}{|p{0.3\linewidth}|p{0.6\linewidth}|}
		\hline % Asegúrate de que esta línea esté aquí
		\rowcolor{grayshade} 
		\textbf{Objetivo 5} & \textbf{Desarrollo e implementación de la aplicación} \\
		\hline % Duplicar esta línea después de \rowcolor puede ayudar
		\textbf{Descripción} & Analizar, desarrollar y probar la aplicación con el objetivo de conseguir los mejores resultados.  \\
		\hline
	\end{tabular}
	\caption{Objetivo 5}
\end{table}

% Tabla para el Objetivo 6
\begin{table}[htb!]
	\centering % Centra la tabla en la página
	\begin{tabular}{|p{0.3\linewidth}|p{0.6\linewidth}|}
		\hline % Asegúrate de que esta línea esté aquí
		\rowcolor{grayshade} 
		\textbf{Objetivo 6} & \textbf{Validación de la aplicación por usuarios reales} \\
		\hline % Duplicar esta línea después de \rowcolor puede ayudar
		\textbf{Descripción} & Poner a prueba la aplicación en un entorno real para ver si cumple las expectativas del usuario.  \\
		\hline
	\end{tabular}
	\caption{Objetivo 6}
\end{table}

\section{Estructura del documento}
\label{sec:document_structure}
Esta sección se ofrece una visión de la estructura y organización del documento. Se compone de las siguientes partes: 

\begin{itemize}
	\item \textbf{Resumen:} Un resumen donde se exponen las principales funcionalidades del proyecto y sus objetivos. 
	
	\item \textbf{Introducción:} Encontramos tres secciones iniciales: 
	\begin{itemize}
		\item La motivación del proyecto.
		\item La estructura general del documento.
		\item Los objetivos que se pretenden cumplir durante el desarrollo del proyecto.
	\end{itemize}
	
	\item \textbf{Planificación:} Especificación de la planificación del proyecto, donde se muestran las fases y la estimación presupuestaria del mismo. 
	
	\item \textbf{Análisis:} Especificación de los requisitos del sistema y los casos de uso. 
	
	\item \textbf{Diseño:} Descripción de la estructura y el diseño de las clases necesarias para el programa y la interfaz de usuario.
	
	\item \textbf{Implementación:} Aspectos a destacar de la implementación del proyecto, problemas encontrados y soluciones aplicadas. Podemos distinguir tres subsecciones: 
	\begin{itemize}
		\item Herramientas utilizadas y justificación de su elección.
		\item Implementaciones clave durante las fases de desarrollo.
		\item Diseño de la interfaz de usuario.
	\end{itemize}
	
	\item \textbf{Conclusiones:} Resumen de los resultados obtenidos y contemplación de posibles direcciones futuras para el proyecto.
	
	\item \textbf{Bibliografía:} Recopilación de todas las fuentes de información utilizadas durante la realización del proyecto.
	
	\item \textbf{Anexos:} Incluye documentos adicionales de valor para el proyecto, como un glosario de términos y acrónimos relevantes.
\end{itemize}
\newpage






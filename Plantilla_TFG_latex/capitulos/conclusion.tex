\chapter{Conclusiones y trabajos futuros}
\label{chap:conclusion}

\section{Conclusiones}

Al finalizar el proyecto de diseño e implementación de una aplicación para la gestión de una tienda minorista, se han alcanzado diversos objetivos importantes que nos permiten evaluar el éxito del trabajo realizado. A continuación se expondrán los objetivos que se han completado y las acciones que se han llevado a cabo para completarlos: 

\begin{itemize}
	\item \textbf{Objetivo 1 - Análisis de los comercios minoristas: } Este objetivo se ha llevado a cabo en el capítulo 3 de este documento. Gracias a este análisis, se ha podido comprender mejor los requisitos necesarios para desarrollar la aplicación. 
	\item \textbf{Objetivo 2 - Estudio de aplicaciones similares: } Este objetivo se ha llevado a cabo en el apartado 3.6 de este documento. Mediante este estudio, se ha podido observar las funcionalidades necesarias que debía de tener la aplicación. Además, hemos podido encontrar funcionalidades innovadoras que no existieran en el mercado actual. 
	\item \textbf{Objetivo 3 - Análisis de usabilidad y accesibilidad: } Este análisis se ha llevado a cabo en el apartado 3.4 de este documento. Gracias a este análisis, hemos podido desarrollar una interfaz gráfica usable y accesible. De esta forma, un mayor número de usuarios podrán utilizar la aplicación desarrollada. 
	\item \textbf{Objetivo 4 - Análisis de tecnologías a utilizar: } Este objetivo se ha desarrollado en el capítulo 4 de este documento. Este análisis ha permitido obtener las mejores tecnologías para el desarrollo de la aplicación. En ese capítulo se encuentra la justificación de dicha elección. 
	\item \textbf{Objetivo 5 - Desarrollo e implementación de la aplicación: } Este objetivo se ha llevado a cabo en los capítulos 5, 6, 7 y 8 de este documento. La aplicación se ha desarrollado mediante iteraciones que permitían mejoras incrementales. Tras la última iteración, se consiguió un entregable que cumplía con los requisitos planteados al inicio del proyecto y aquellos resultantes de las valoraciones de las iteraciones. 
	\item \textbf{Objetivo 6 - Validación de la aplicación por usuarios reales: } Este objetivo no se ha conseguido cumplir en su totalidad. Tras cada iteración, el resultado era valorado por un usuario real. Sin embargo, no se ha llegado a poner en un entorno real. Por ello, deberá de hacerse como trabajo futuro. 
\end{itemize}

Una vez presentados los objetivos que se han cumplido a lo largo del proyecto, se presentan los principales logros conseguidos: 

\begin{itemize}
	\item \textbf{Funcionalidades certeras: } Al realizar una investigación sobre los negocios minoristas, he podido entender mejor cuáles son las necesidades a las que se enfrentan y poder definir unos requisitos más acertados a la hora del diseño de la aplicación. Además, tras cada iteración, se obtenía una validación de usuario que permitía saber si la aplicación se estaba desarrollando adecuadamente. 
	\item \textbf{Novedad de la aplicación: } A diferencia de otras soluciones existentes en el mercado, la nueva aplicación desarrollada introduce funcionalidades innovadoras que ofrecen servicios únicos que hasta ahora no estaban disponibles. Una de las funcionalidades que ninguna otra aplicación del mercado contemplaba, es la introducción de un sistema de préstamos. 
	\item \textbf{Compatibilidad con Android e IOS: } Al desarrollar la aplicación en Flutter con el lenguaje de programación Dart, la aplicación se puede adaptar a ambos sistemas operativos con facilidad. Esto permite que la aplicación pueda ser utilizada por un mayor número de usuarios.  
	\item \textbf{Accesibilidad y usabilidad: } Se han tenido en cuenta los principios de accesibilidad y usabilidad para desarrollar la aplicación, obteniendo como resultado una aplicación sencilla de usar para una gran cantidad de usuarios. 
\end{itemize}

\section{Trabajos futuros}

Durante el desarrollo de la aplicación, se han realizado una serie de pruebas de funcionalidad que aseguran unos resultados válidos. Sin embargo, para continuar con el desarrollo y mejora de la aplicación, se propone como trabajo futuro la introducción de la misma en un entorno real que permita evaluar su funcionamiento en condiciones auténticas. Este proceso no solo busca identificar posibles fallos o áreas de mejora, sino también adaptar la aplicación a las necesidades y expectativas de los usuarios finales. Para ello se deberá: 

\begin{itemize}
	\item \textbf{Seleccionar un entorno de prueba: } Se deberá de elegir un entorno de prueba que refleje de forma precisa el contexto en el que se espera que la aplicación sea utilizada. Por tanto, deberá de ser una pequeña tienda minorista sin informatizar con una pequeña clientela. 
	\item \textbf{Involucramiento de usuarios reales: } Durante esta fase, se pondrá en marcha la aplicación en el entorno previamente seleccionado y se observará en detalle la interacción de dichos usuarios con la aplicación. 
	\item \textbf{Recolecta de datos y análisis: } Con los datos obtenidos de la fase anterior, se harán propuestas de mejora para que el funcionamiento de la aplicación se adapte al entorno. 
	\item \textbf{Mejora de la implementación: } Esta etapa trata de aplicar las mejoras que se han diseñado en el código de la aplicación. Es decir, se harán físicos los cambios planteados de forma teórica. 
	\item \textbf{Lanzamiento final: } Finalmente, tras someter la aplicación a unas pruebas con usuarios reales, se consigue un resultado más ajustado a las necesidades reales. 
\end{itemize}

Tras completar esta valoración con  usuarios finales, el resultado de la aplicación quedaría mejorado. 

Al finalizar la implementación de la aplicación, se deberá hacer un manual de usuario para explicar todas las funcionalidades de la aplicación. De esta forma, será mas sencillo el entendimiento del funcionamiento de la aplicación. 